%----------------------------------------------------------------------------------------
%	PACKAGES AND OTHER DOCUMENT CONFIGURATIONS
%----------------------------------------------------------------------------------------

\documentclass[
	10pt,
]{style} % Use the style class

\usepackage{libertine}
\usepackage{hyperref} % Use the EB Garamond font
\usepackage{enumitem} % Required for customizing lists

\usepackage{setspace}
\setstretch{0.99} % Réduit l'interligne à 95% de la normale

%------------------------------------------------

\name{Eyléa Piouceau}

\address{+33 7 81 61 89 65} % Contact information

\address{\href{mailto:eylea.piouceau@ens-paris-saclay.frr}{eylea.piouceau@ens-paris-saclay.fr}}

%----------------------------------------------------------------------------------------

\begin{document}

\vspace{-0.25em}

\begin{center}
	6 rue Jalna, 78370 Plaisir, France \\
\end{center}

\begin{center}
	\textbf{Status: Student at École Normale Supérieure Paris-Saclay with civil servant status} \\
	\textbf{Objective: Application for an internship next summer in Biology}
\end{center}

%----------------------------------------------------------------------------------------
%	EDUCATION SECTION
%----------------------------------------------------------------------------------------

\begin{rSection}{Education}

	\textbf{École Normale Supérieure Paris-Saclay} \hfill 2024 - Present \\
	Master 1 Biologie-Santé \hfill \textit{Gif-sur-Yvette, France} \\
	\textit{First year of an MSc in Life Sciences and Health at University Paris-Saclay} \\
	\textit{Second year of the specific ENS Paris-Saclay diploma, a 4-year reinforced training} \\
	Admitted to the ENS Paris-Saclay after a highly competitive entrance exam, status of civil servant obtained

	\vspace{0.5mm}

	\textbf{Université de Versailles Saint-Quentin-en-Yvelines} \hfill 2021 - 2024 \\
	Licence double-diplôme Chimie, Sciences de la vie (240 ECTS) \hfill \textit{Versailles, France} \\
	\textit{Dual Bachelor's degree in Chemistry and Life Sciences} \\
	Valedictorian over the 3 years with an overall average of 18,5/20

	\vspace{0.5mm}

	\textbf{Lycée Jean Vilar} \hfill 2018 - 2021 \\
	Baccalauréat mention très bien avec les félicitations du jury \hfill \textit{Plaisir, France} \\
	\textit{Equivalent to A-levels in science subjects obtained with highest honours} \\
	Overall average : 18,4/20
\end{rSection}

%----------------------------------------------------------------------------------------
%	PROFESSIONAL EXPERIENCE SECTION
%----------------------------------------------------------------------------------------

\begin{rSection}{Professional Experience}

	\begin{rSubsection}{Research intern at IBPC (Institut de Biologie Physico-Chimique)}{June 2023 - July 2023}{2-month internship in the 'Microbial Genetic Expression' Unit (UMR8261 CNRS), directed by Ciarán Condon.}{Paris, France}
		\item As a Research Intern within the ``RNA control of gene expression'' team led by Maude Guillier, I studied the regulation of gene expression in \textit{Escherichia coli.}
		\item For the project, I needed to carry out several laboratory techniques such as northern blot, bacterial transformation, site-directed mutagenesis and $\beta$-galactosidase activity assays.
		\item I had the opportunity to develop my analytical thinking skills and gain a deeper understanding of the scientific approach.
	\end{rSubsection}

%------------------------------------------------

	\begin{rSubsection}{Part-time student worker - Course reviewer, Université de Versailles Saint-Quentin-en-Yvelines}{May 2023 - June 2023}{Course reviewer for the HILISIT project (Hybridation en LIcence ScIenTifique)}{Versailles, France}
		\item I was selected to participate in a project where I reviewed and corrected course materials written by university lecturers and researchers. The courses were centered on the core topics of molecular biology and were designed to serve as online educational resources for undergraduate students.
		\item This project allowed me to reflect on course structure and adapt concepts according to the students' proficiency levels.
	\end{rSubsection}

\end{rSection}

%----------------------------------------------------------------------------------------
%	TECHNICAL AND LANGUAGES SKILLS SECTION
%----------------------------------------------------------------------------------------

\begin{rSection}{Technical and Language Skills}

	\textbf{Foreign Languages}
	\begin{itemize}[noitemsep, topsep=0pt, left=0pt]
		\item I have a good command of English, supported by my score of 7.5 (equivalent to a C1) on the IELTS examination.
		\item I am currently preparing for the SWAP (Scientific Writing Assessment Program) certification in Scientific English.
		\item I am a native speaker of French.
	\end{itemize}
	\textbf{Programming Languages}
	\begin{itemize}[noitemsep, topsep=0pt, left=0pt]
		\item I am a proficient user of all Microsoft Office packages and I am experienced in using PubMed to conduct bibliographic research.
		\item I have a good working knowledge of Python and I am familiar with R.
	\end{itemize}

\end{rSection}


%----------------------------------------------------------------------------------------
%	PERSONAL INTERESTS SECTION
%----------------------------------------------------------------------------------------

\begin{rSection}{Personal Interests}
	\vspace{-1em}
	\begin{rSubsection}{}{}{}{}
		\item I have been engaged in sports since childhood, initially with classical ballet for 10 years, followed by club volleyball for the past 5 years, along with Pilates and Yoga. Team sports have helped me develop team-working and communication skills, while sports in general have taught me perseverance and discipline.
		\item I am a volunteer at a neighborhood council, where during the summer of 2024, I assisted primary school students with their homework twice a week for two months. This experience taught me how to adapt to different profiles and develop a pedagogical approach to support the students.
	\end{rSubsection}

\end{rSection}


%----------------------------------------------------------------------------------------
%	REFERENCES SECTION
%----------------------------------------------------------------------------------------

\begin{rReferences}
	\vspace{-0.5em}
    \item \textit{Academic Reference:} Uriel Hazan, research professor at ENS Paris-Saclay $\diamond$ \href{mailto:uriel.hazan@ens-paris-saclay.fr}{uriel.hazan@ens-paris-saclay.fr}
	\item \textit{Academic Reference:} Clemence Richetta, associate professor responsible for 2nd year students at ENS Paris-Saclay $\diamond$ \href{mailto:clemence.richetta@ens-paris-saclay.fr}{clemence.richetta@ens-paris-saclay.fr}
	\item \textit{Scientific Reference:} Maude Guillier, research director at IBPC $\diamond$ \href{mailto:maude.guillier@ibpc.fr}{maude.guillier@ibpc.fr}
\end{rReferences}

\end{document}

