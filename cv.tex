%----------------------------------------------------------------------------------------
%	PACKAGES AND OTHER DOCUMENT CONFIGURATIONS
%----------------------------------------------------------------------------------------

\documentclass[
	10pt,
]{style} % Use the style class

\usepackage{libertine}
\usepackage{hyperref} % Use the EB Garamond font
\usepackage{enumitem} % Required for customizing lists

\usepackage{setspace}
\setstretch{0.99} % Réduit l'interligne à 95% de la normale

%------------------------------------------------

\name{Eyléa Piouceau}

\address{+33 7 81 61 89 65} % Contact information

\address{\href{mailto:eylea.piouceau@ens-paris-saclay.frr}{eylea.piouceau@ens-paris-saclay.fr}}

%----------------------------------------------------------------------------------------

\renewenvironment{rReferences}
{\subsubsection*{Références}
	\begin{list}{} % environnement list sans puces
	{\setlength{\leftmargin}{2em}   % décalage à gauche
	\setlength{\labelwidth}{0pt}
	\setlength{\itemindent}{0pt}
	\setlength{\labelsep}{0pt}
	\setlength{\parsep}{0pt}
	\setlength{\topsep}{2em}
	\setlength{\itemsep}{0.5em}}}   % espace entre les items
	{\end{list}}


\begin{document}

\vspace{-0.25em}

\begin{center}
	6 rue Jalna, 78370 Plaisir, France \\
\end{center}

\begin{center}
	\textit{Statut : Élève fonctionnaire stagiaire à l'École Normale Supérieure Paris-Saclay} \\
	\textbf{Objet : Candidature pour un stage d'été en Biologie} \\
\end{center}

%----------------------------------------------------------------------------------------
%	EDUCATION SECTION
%----------------------------------------------------------------------------------------

\begin{rSection}{Formation}

	\textbf{École Normale Supérieure Paris-Saclay} \hfill 2025 - Présent \\
	Master 2 Formation à l'Enseignement Supérieur en Sciences du Vivant (Agrégation BBB)  \hfill \textit{Gif-sur-Yvette, France} \\
	\textit{Troisième année du diplôme de l'ENS Paris-Saclay}

	\vspace{0.5mm}

	\textbf{École Normale Supérieure Paris-Saclay} \hfill 2024 - 2025 \\
	Admise à l'ENS Paris-Saclay via le second concours en tant qu'élève fonctionnaire stagiaire \\
	Master 1 Biologie-Santé voie Boris Ephrussi, plateforme Microbiologie  \hfill \textit{Gif-sur-Yvette, France} \\
	\textit{Deuxième année du diplôme de l'ENS Paris-Saclay} \\
	Major de promotion avec une moyenne générale de 17,7/20
	\vspace{0.5mm}

	\textbf{Université de Versailles Saint-Quentin-en-Yvelines} \hfill 2021 - 2024 \\
	Licence double-diplôme Chimie, Sciences de la vie (240 ECTS) \hfill \textit{Versailles, France} \\
	Major de promotion avec une moyenne générale de 18,5/20

	\vspace{0.5mm}

	\textbf{Lycée Jean Vilar} \hfill 2018 - 2021 \\
	Baccalauréat mention très bien avec les félicitations du jury \hfill \textit{Plaisir, France} \\
	Moyenne générale : 18,4/20
\end{rSection}

%----------------------------------------------------------------------------------------
%	PROFESSIONAL EXPERIENCE SECTION
%----------------------------------------------------------------------------------------

\begin{rSection}{Expériences professionnelles}

	\begin{rSubsection}{Stagiaire à Hubrecht Institute}{June 2025 - July 2025}{Stage de 6 semaines dans l'équipe 'Gene Expression Dynamics' dirigée par Marvin Tanenbaum.}{Utrecht, Pays-Bas}
		\item En tant que chercheur stagiaire, j'ai étudié l'impact de l'optimalité des codons sur la dynamique de l’élongation de la traduction par imagerie en cellule vivante d’ARN circulaires.
		\item J’ai réalisé des acquisitions d’images fluorescentes de cellules humaines par microscopie confocale à disque rotatif et analysé les données avec le logiciel TransTrack.
	\end{rSubsection}

	\begin{rSubsection}{Stagiaire à l'IBPC (Institut de Biologie Physico-Chimique)}{Juin 2023 - Juillet 2023}{Stage de 2 mois dans l'unité 'Expression Génétique Microbienne (UMR8261 CNRS), dirigée par Ciarán Condon.}{Paris, France}
		\item En tant que chercheur stagiaire au sein de l'équipe ``Contrôle de l'expression génétique par les ARNs'' dirigée par Maude Guillier, j'ai étudié la régulation de l'expression des gènes chez \textit{Escherichia coli.}
		\item J'ai utilisé différentes techniques expérimentales telles que la transformation bactérienne, la PCR, le northern blot et des mesures de l’activité $\beta$-galactosidase.

	\end{rSubsection}

%------------------------------------------------

	\begin{rSubsection}{Étudiant vacataire - Relecture de cours, Université de Versailles Saint-Quentin-en-Yvelines}{Mai 2023 - Juin 2023}{Correction de cours dans le cadre duprojet HILISIT (Hybridation en LIcence ScIenTifique)}{Versailles, France}
		\item J'ai été sélectionnée pour relier et corriger des cours rédigés par des enseignants-chercheurs sur les principaux chapitres de biologie moléculaire de licence, ayant pour vocation de devenir des ressources en ligne.
	\end{rSubsection}

\end{rSection}

%----------------------------------------------------------------------------------------
%	TECHNICAL AND LANGUAGES SKILLS SECTION
%----------------------------------------------------------------------------------------

\begin{rSection}{Compétences linguistiques et techniques}

	\textbf{Langues}
	\begin{itemize}[noitemsep, topsep=0pt, left=0pt]
		\item Anglais : niveau C1 (score de 7.5/9 à l'IELTS). Certifiée en anglais sientifique (SWAP - Scientific Writing Assessment Program).
		\item Français : langue maternelle
	\end{itemize}
	\textbf{Langages de programmation et outils informatiques}
	\begin{itemize}[noitemsep, topsep=0pt, left=0pt]
		\item Compétente dans la recherche bibliographique via Pubmed.
		\item Bonne maîtrise de Python et connaissances de base en R
	\end{itemize}

\end{rSection}


%----------------------------------------------------------------------------------------
%	PERSONAL INTERESTS SECTION
%----------------------------------------------------------------------------------------

\begin{rSection}{Intérêts personnels et bénévolat}
	\vspace{-1em}
	\begin{rSubsection}{}{}{}{}
		\item Sport (vélo, course à pied, pilates et yoga), Musique (piano, ukulélé et chant), Lecture
		\item Je suis bénévole au Conseil de quartier du Valibout au sein duquel j'ai encadré du soutien scolaire pour des élèves de l'école élémentaire.
	\end{rSubsection}

\end{rSection}


%----------------------------------------------------------------------------------------
%	REFERENCES SECTION
%----------------------------------------------------------------------------------------

\begin{rReferences}
	%\vspace{-0.5em}
	\item \textit{Référent académique :} Uriel Hazan, professeur à l'ENS Paris-Saclay $\diamond$ \href{mailto:uriel.hazan@ens-paris-saclay.fr}{uriel.hazan@ens-paris-saclay.fr}
	\item \textit{Référente académique :} Clémence Richetta, Maître de Conférences à l'ENS Paris-Saclay $\diamond$ \href{mailto:clemence.richetta@ens-paris-saclay.fr}{clemence.richetta@ens-paris-saclay.fr}
	\item \textit{Référente scientifique :} Maude Guillier, directrice de recherche à l'IBPC $\diamond$ \href{mailto:maude.guillier@ibpc.fr}{maude.guillier@ibpc.fr}
\end{rReferences}

\end{document}

